\documentclass{article}
\usepackage{geometry}
\usepackage[utf8]{inputenc}
\usepackage{hyperref}
\usepackage[table,xcdraw]{xcolor}
\usepackage{enumerate}
\usepackage{enumitem}
\usepackage{background}
\usepackage{graphicx}
\usepackage[a4paper, margin=0.9in]{geometry}
\usepackage{tikz}
\usetikzlibrary{calc}
\backgroundsetup{contents=\includegraphics{background.jpeg},opacity=0.7,angle=0,scale=1}
\geometry{left=1in, right=1in, top=1in, bottom=1in}
\usepackage{titling}

\begin{document}


\begin{center}
    \huge\textbf{MAULANA ABUL KALAM AZAD UNIVERSITY OF TECHNOLOGY}\\

\end{center}
\begin{figure}[h!]
    \centering
    \includegraphics[width=0.3\linewidth]{makaut.png}
\end{figure}
\date{\today} 
\begin{center}
    \section*{\textbf{\underline{Software Tools and Technology (Lab Notbook)}}}
    \vspace{0.4cm}
\end{center}
\begin{center}

\vspace{0.2 cm}
\renewcommand{\arraystretch}{2}
\hspace*{0.08in}
\begin{tabular}{ |c|c|c| }
\hline
\multicolumn{3}{|c|}{\Large \textbf{\textit{Group 12}}} \\
\hline
NAME & ROLL NO.& DEPARTMENT \\
\hline
Sk Shoaib Akhter [LEAD]& 30001223043& BCA \\
\hline
Urjjaswi Paul& 30059223024& Bsc in Forensic Science\\
\hline
Pranjal Paul& 30059223023& Bsc in Forensic Science\\
\hline
Rupkatha Bhowmik& 30059223009& Bsc in Forensic Science\\
\hline
Alankrita Ghosh& 30059223010& Bsc in Forensic Science\\
\hline
\end{tabular}

\end{center}
\newpage

\vspace{3cm} 
\begin{center}
    \Huge \textbf{\textcolor{black}{\underline{Table of Contents}}} 
\end{center}

\vspace{2cm} 

\begin{center}
\begin{tabular}{|>{\centering\arraybackslash}m{2cm}|m{8cm}|}
\hline
\textbf{Sl. No.} & \textbf{Questions} \\
\hline
1 & Introduction to GitHub and GitHub Desktop version installation. \\

\hline
2 & Enumerate ABC Format and Roman Number in LaTeX\\

\hline
3 & Building a C program for a calculator in the local repository, committing, and publishing it as a public repository.\\

\hline
4 & How to create matrix in LaTeX.\\

\hline
5 & Rename the button from "Submit" to "Chin Tapak Dum Dum." and create a pull request\\
\hline
\end{tabular}
\end{center}



\newpage 


\begin{center}
    \Large{\textbf{\underline{Lab -1 by Shoaib}}}
    \end{center}
    \vspace{1cm}
   \begin{center}
    \textbf{\underline{ Introduction to GitHub and GitHub installation}}
\end{center}
\begin{figure}[h!]
    \centering
    \includegraphics[width=0.25\linewidth]{Github.png}
\end{figure}
\begin{center}
    \section*{\textbf{\underline{GitHub}}}
\end{center}


\large{GitHub is a cloud-based platform specifically designed for developers to collaborate on code and manage software projects using Git, a widely used version control system. At its core, GitHub provides a space called repositories where code is stored, allowing users to track changes over time and maintain a complete history of all updates made to the project. Through GitHub, developers can work on different features or versions of a project simultaneously by creating separate branches, which can later be merged back into the main codebase using pull requests, ensuring smooth integration of changes.}
\vspace{1 cm}
\begin{center}
    \section*{\textbf{\underline{Installation}}}
\end{center}

\large{Installing GitHub involves a few simple steps to get started with managing and collaborating on code. First, download and install Git, the version control system that GitHub uses, from the official Git website. For Windows users, run the downloaded installer and follow the prompts. On macOS, you can install Git through the terminal if it’s not already present. For Linux users, Git can be installed via the terminal with package management commands. Once Git is set up, configure it with your GitHub username and email to link your commits to your GitHub account. If you prefer a graphical interface, you can also download GitHub Desktop from its official website, which makes it easier to handle repositories, make commits, and manage branches without using the command line. After installation, sign in with your GitHub credentials or create a new account if necessary.}
\newpage

\begin{center}
     \textbf{\underline{\large{Lab-2 by Pranjal}}}
     \end{center}
     \vspace{0.5cm}
\begin{center}
     \textbf{\large{\underline{Enumerate ABC Format and Roman Number in LaTeX}}}
 \end{center}

\begin{center}
    \section*{\underline{\textbf{\large{Introduction}}}}
\end{center}


\large{To create enumerated lists in LaTeX using both alphabetic (ABC) format and Roman numeral format, you can use the \texttt{enumerate} environment. LaTeX allows customization of list labels either by using the \texttt{enumerate} package (basic customization) or the \texttt{enumitem} package (more advanced customization). Below is a detailed explanation and corresponding LaTeX code.}

\subsection*{1.\underline{Alphabetic Format (ABC):}}
\begin{itemize}
    \item For alphabetic enumeration (A, B, C...), you can use either the \texttt{enumerate} package or the \texttt{enumitem} package.
  \end{itemize}

\subsection*{2. \underline{Roman Numerals:}}
\begin{itemize}
    \item For Roman numeral enumeration (I, II, III...), you can use \texttt{[I.]} with the \texttt{enumerate} package, or \texttt{label=\Roman*.} with the \texttt{enumitem} package.
\end{itemize}

\section*{\underline{Example}}

Here’s the LaTeX code that implements both alphabetic and Roman numeral enumerations:

\begin{verbatim}
\documentclass{article}

% Using the enumerate package for basic list formatting
\usepackage{enumerate}

% Using the enumitem package for advanced list formatting
\usepackage{enumitem}

\begin{document}

\section*{Enumerated Lists: ABC Format and Roman Numerals}

\subsection*{Using the enumerate Package}

\subsubsection*{Alphabetic Format (ABC)}

% Alphabetic enumeration using enumerate package
\begin{enumerate}[A.]
  \item First item
  \item Second item
  \item Third item
\end{enumerate}

\subsubsection*{Roman Numerals}

% Roman numeral enumeration using enumerate package
\begin{enumerate}[I.]
  \item First item
  \item Second item
  \item Third item
\end{enumerate}

\subsection*{Using the enumitem Package}

\subsubsection*{Alphabetic Format (ABC)}

% Alphabetic enumeration using enumitem package
\begin{enumerate}[label=\Alph*.]
  \item First item
  \item Second item
  \item Third item
\end{enumerate}

\subsubsection*{Roman Numerals}

% Roman numeral enumeration using enumitem package
\begin{enumerate}[label=\Roman*.]
  \item First item
  \item Second item
  \item Third item
\end{enumerate}

\end{document}
\end{verbatim}
\section*{\underline{Explanation}}

\subsection*{1. Alphabetic Format (ABC):}
\begin{itemize}
    \item \textbf{Using the enumerate package:} In the \texttt{enumerate} environment, you can specify the list type as \texttt{[A.]} to generate an alphabetic list (A, B, C...).
    \item \textbf{Using the enumitem package:} The \texttt{enumitem} package allows more flexibility by setting the label format as \texttt{label=\Alph*.}.
\end{itemize}
\subsection*{2. Roman Numerals:}
\begin{itemize}
    \item \textbf{Using the enumerate package:} For Roman numerals, you use \texttt{[I.]} to create the list items as I, II, III...
    \item \textbf{Using the enumitem package:} You can customize the label with \texttt{label=\Roman*.} to achieve the same effect with more control over spacing and style.
\end{itemize}

\section*{\underline{Output:}}

The compiled LaTeX document will display two types of lists (ABC and Roman numerals) using both \texttt{enumerate} and \texttt{enumitem} packages.
\newpage 

\begin{center}
    \Large{\textbf{\underline{Lab-3 by Rupkatha}}}
\end{center}
\section{\underline{Calculator Program using C}}
\subsection{\underline{Objective}}
\LARGE{The objective of this lab is to develop a basic calculator program using the C programming language. The calculator will perform simple arithmetic operations like addition, subtraction, multiplication, and division based on user input.}

\subsection{\underline{Program Overview}}
The calculator program is designed to:
\begin{itemize}
    \item Accept two numbers from the user.
    \item Prompt the user to select an arithmetic operation (Addition, Subtraction, Multiplication, Division).
    \item Perform the selected operation.
    \item Display the result of the operation to the user.
\end{itemize}

The program includes error handling to manage division by zero and other invalid inputs.

\subsection{\underline{Code Implementation}}

\begin{verbatim}
#include<stdio.h>
#include<conio.h>
void main(){
    float a,b,c;
    char ch;
        printf("Enter the first number : ");
        scanf("%f",&a);
            printf("Enter user choice operation : ");
            scanf(" %c",&ch);
        printf("Enter the second number : ");
        scanf("%f",&b);
    switch(ch){
        case'+':c=a+b;
            printf("Output is : %f",c);
        break;
        case'-':c=a-b;
            printf("Output is : %f",c);
        break;
        case'*':c=a*b;
            printf("Output is : %f",c);
        break;
        case'/':c=a/b;
            printf("Output is : %f",c);
        break;
            default:printf("Invalid operation");
        break;
    } getch();
}
\end{verbatim}
\subsection{\underline{Compiling and Running the Program}}
To compile and run the calculator program:
\begin{enumerate}
    \item Open a terminal or command prompt.
    \item Navigate to the directory where the C file is located.
    \item Compile the program using a C compiler (e.g., GCC):
    \begin{verbatim}
    gcc calculator.c -o calculator
    \end{verbatim}
    \item Run the compiled program:
    \begin{verbatim}
    ./calculator
    \end{verbatim}
\end{enumerate}
\subsection{\underline{Adding the Calculator Program to GitHub Repository}}
To add this calculator program to a GitHub repository, follow these steps:

\subsubsection{Step 1: Initialize a Local Git Repository}
\begin{enumerate}
    \item Open the terminal and navigate to the directory where your \texttt{calculator.c} file is located.
    \item If you haven't already, initialize a Git repository in that directory:
    \begin{verbatim}
    git init
    \end{verbatim}
    This command creates a new Git repository in the current directory.
\end{enumerate}


\subsubsection{Step 2: Add the File to the Repository}
\begin{enumerate}
    \item Add the \texttt{calculator.c} file to the staging area:
    \begin{verbatim}
    git add calculator.c
    \end{verbatim}
    This command stages the file, indicating that you want to include it in the next commit.
\end{enumerate}

\subsubsection{Step 3: Commit the Changes}
\begin{enumerate}
    \item Commit the file to the repository with a meaningful message:
    \begin{verbatim}
    git commit -m "Add calculator program in C"
    \end{verbatim}
\end{enumerate}
\subsubsection{Step 4: Push the Changes to GitHub}
\begin{enumerate}
    \item Link your local repository to a remote GitHub repository:
    \small{
    \begin{verbatim}
    git remote add origin https://github.com/yourusername/your-repo-name.git
    \end{verbatim} }
    \item Push the changes to the GitHub repository:
    \begin{verbatim}
    git push -u origin master
    \end{verbatim}
\end{enumerate}

\subsubsection{Step 5: Verify the Upload}
\begin{enumerate}
    \item Go to your GitHub repository URL in a web browser.
    \item Verify that the \texttt{calculator.c} file is listed and accessible in the repository.
\end{enumerate}
\end{document}
